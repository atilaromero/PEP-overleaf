\chapter{Proposal}
\section{Objectives}
The main objective of the proposed work is to verify if machine learning techniques could be applied to perform data-carving, answering the following initial questions:

\begin{itemize}
  \item Is this feasible?
  \item How each technique compare to each other, concerning training performance and results quality?
  \item How they compare to existing solutions?
  \item Neural networks are more advantageous than other machine learning approaches?
  \item Usage of LSTM neural networks helps to interpret internal structures of files?
  \item Which framework to use?
  \item Which alphabet to use?
\end{itemize}

\section{Methodology}
In order to compare solutions, a methodology of comparison must be defined, specifying metrics and datasets.
% possiveis metricas
% datasets existentes
%  exemplos
% datasets criados
%  exemplos


Classic data-carving tools should be used to process those datasets, to create a research baseline.
% exemplos de ferramentas que poderao ser usadas 

\section{Infrastructure}
Before starting the tests, an appropriate environment must be built. This requires evaluation of available machine learning tools and frameworks. Example of some of the available options include:
% exemplos de frameworks e ferramentas

During evaluation of those tools, while building the environment, some of the most simple solution alternatives can begin to be tested, probably training binary classification of file types, using a simple dataset consisting of small files. This will allow to both compare frameworks and to begin the tests.

\section{Experiments}
Following the environment preparation, the next step should be to compare solutions, starting with the most simple solutions first, and increasing complexity next.

Initially, those are the planned networks and scenarios to be tested:
% inserir tabela
% redes neurais simples
% redes neurais convolucionais
% redes neurais LSTM e BLSTM
% dataset: arquivos
% dataset: setores

It would be positive to include tests with approaches other than neural networks, but it may happen that the construction of the necessary environment and implementation of the algorithms doubles the amount of work involved. For that reason this step will be left as desired but optional.

\section{Schedule}
