\section{Challenges}

Each of the steps cited by \cite{ali_review_2018} for the data carving process deals with a main challenge. The identification step is responsible for classifying the file type. The quantity and diversity of file types, together with accuracy and precision of the results, are the main challenges in identification. Validation also deals with classification, but it is a complementary step to the previous one to reduce false positives, often using a different technique. For that reason, validation challenges are similar to identification ones. Reassembling main challenge is fragmentation.

In the current proposal, a new type of challenge is introduced. Instead of using previous knowledge of the file structure to improve carving results, would be possible to do the inverse and use insights from the carving process to reveal structures in the file?

One of the most simple structures a file can have is a fixed size field, such as 32 bit a unsigned integer representing a datetime value. For that type of field, the insight may come in the form of a expected range. Still using the datetime example, a possible outcome would be the observation that certain kind of file always presents that field value inside some range, that coincides with a range often observed in datetime fields. That does not prove the unknown field to be a datetime, but suggests that direction.

Few structures are so simple as a group of fixed sized fields. It is very common, for example, to use a field to specify the length of the next field. Another type of complexity increase occurs when the value of a field establishes which specification should be used in the remaining of the file, changing which fields should be expected next.

The greater the complexity of a unknown file structure is, the more difficult it is to unravel its specification, but also the more useful it is to count with tools that automatize that task. Otherwise, the only option is to manually write the specification or the parser, possibly relying in reverse engineering techniques.

The direct utility of the discovery of file type structures is the extraction of values from its fields. This information has value for itself, but could also be used to improve validation and even reassembling. 
