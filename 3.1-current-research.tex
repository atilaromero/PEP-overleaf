\section{Current research}
To assess the current research status on data-carving machine learning technologies, four digital libraries were consulted, ACM (https://dl.acm.org/), IEEE (https://ieeexplore.ieee.org/), Scopus (https://www.scopus.com/) and Springer Link (https://link.springer.com/), using the search string 
{\texttt{(``data carving'' AND ``machine learning'')}}. Using ``data-carving'' or ``data carving'' (with or without hyphen) did not change any of the libraries results. Later the search was repeated using the terms {\texttt{(``data carving'' AND ``neural networks'')}}.

This initial approach brought few results, only four of them considered relevant: \cite{alamri_taxonomy_2014}, 
\cite{ali_review_2018}, \cite{sportiello_context-based_2012}, and \cite{beebe_sceadan:_2013}. But using citations and references of this four works, the initial results were expanded and gave a more comprehensive view of current research on using machine learning techniques to perform data carving.

The first work found mentioning the usage of a neural network in data carving was \cite{amirani_new_2008}, in 2008. It uses Principal Component Analysis (PCA) as input for a 5 layer feed-forward auto-associative unsupervised neural network to do feature extraction and a 3 layer Multi Layer Perceptron (MLP) to perform classification.

In 2014, \cite{alamri_taxonomy_2014} created an taxonomy of file type identification techniques, grouped by the following broad categories: statistical learning, frequency distribution, statistical analysis, and detection of file fragments. The statistical learning category is subdivided in supervised and unsupervised leaning. The supervised learning techniques, which are the more relevant for this proposal, are Support Vector Machine (SVM), k-Nearest Neighbor (kNN), and Neural Network (NN). According to this taxonomy study, SVM is used in \cite{ahmed_fast_2011}, \cite{amirani_feature-based_2013}, \cite{beebe_sceadan:_2013}, \cite{fitzgerald_using_2012}, \cite{gopal_statistical_2011}, and \cite{sportiello_context-based_2012}, kNN is used in \cite{ahmed_fast_2011} and \cite{gopal_statistical_2011}, while neural networks are used in \cite{ahmed_fast_2011}, \cite{ahmed_content-based_2010}, \cite{amirani_new_2008}, \cite{amirani_feature-based_2013}, and \cite{penrose_approaches_2013}.

In 2018, \cite{ali_review_2018} reviewed digital forensics methods for JPEG file carving. JPEG is mentioned in this paper as a common focus among data carving studies. Only some of the analyzed studies utilizes some machine learning technique: \cite{xu_reassembling_2009} uses a neural network as a cluster reassembling technique, 
%finish references of \cite{ali_review_2018}

%check references \cite{sportiello_context-based_2012}
%check references \cite{beebe_sceadan:_2013}.

According to \cite{ali_review_2018}, artificial intelligence techniques are found to be not fully utilized in this field.

In 2018, \cite{hiester_file_2018} apparently was the first to utilize a LSTM network to perform file fragment classification.
only 4 types of files
reconstruction

% check \cite{hiester_file_2018}
% describe his work better