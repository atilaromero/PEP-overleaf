\section{Existing tools}
The available data carving tools generally do not take advantage of the latest techniques that research on the field offers, often still relying in header/footer identification and providing limited reassembling capabilities.

% check photorec citation

Available data carving tools are cited by \cite{ali_review_2018}, \cite{qiu_new_2014}, \cite{nadeem_ashraf_forensic_2013}, and \cite{roux_reconstructing_2008}, but the review of \cite{ali_review_2018} is the most comprehensive. Among the listed tools, only Foremost \cite{kendall_notitle_2010}, Scalpel \cite{richard_iii_scalpel:_2005}, and PhotoRec \cite{grenier_photorec_2011} support a wide range of file formats. These three tools rely on header/footer identification.

The amount of work required to support the vast amount of file types in existence is here chosen as an hypothesis for the reason for this discrepancy. Most of the attention paid to the results of data carving research is focused on increasing some statistical measurement of success. While these advances are undoubtedly important, they may not be the kind of research needed to transform these findings into practical tools. If this is the case, then the forensic community would benefit from researches that could make the task of supporting the carving of a new file type easier. Machine learning techniques have potential to achieve that goal, because it can replace the step of manually encoding a structure parser by automatically recognizing patterns in large amounts of data.
