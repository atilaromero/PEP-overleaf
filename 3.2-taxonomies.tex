\section{Data carving taxonomies}

\cite{ali_review_2018} divides the data carving process in three steps:
\begin{description}
    \item [identification] Classifies the file type of individual chunks of data. 
    \item[validation] Includes a list of requirements of a file that are needed for its recovery to be considered successful.
    \item [reassembling] Attempts to reconstruct the original file.
\end{description}

Some of the studies found deal only with the identification step, which has applications of its own, while others also include the challenge of achieving reassembling.

\cite{nadeem_ashraf_forensic_2013} groups the carving techniques in generations, each extending the previous one:
\begin{description}
    \item [first generation] Header-footer based carving. Uses file signatures like magic-bytes, headers and footers to identify beginning and end of a file.
    \item [second generation] Structure based carving, also called ``semantic carving'' or ``deep carving''. Reduces number of false positives by using file structure knowledge to perform validation.
    \item [third generation] Advance reassembling with methods to deal with fragmentation. Tries to infer relationships and order between chunks of data based on content and statistical analysis to reassemble the original file.
\end{description}

For file type detection, which could be mapped to the identification step in the \cite{ali_review_2018} process steps, \cite{amirani_new_2008} cites three categories:
\begin{description}
    \item [extension-based identification] The content of the file is ignored, only its filename extension is used.
    \item [magic bytes-based identification] Magic bytes are signatures, generally a fixed string, usually in the beginning of a file. Is a common strategy that uses header/footer, but not all files adopt it.
    \item [contend-based identification] Identifies the file using some statistical modeling of its content. 
\end{description}

\cite{beebe_sceadan:_2013} identifies three content-based approaches to classify file and data types, also referring only to the identification step of the \cite{ali_review_2018} data carving process division:
\begin{description}
    \item [semantic parsing] Relies on the file structure to identify its type.
    \item [nonsemantic parsing] Searches for strings that are commonly found in specific files.
    \item [machine learning] Uses supervised and unsupervised algorithms, like Support Vector Machine (SVM), k-Nearest Neighbor (kNN), and Neural Networks (NN).
\end{description}

